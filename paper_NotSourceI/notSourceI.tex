
\Figure
%{figures/Orion_SourceI_B6_continuum_r-2.clean0.5mJy.selfcal.ampphase5.image.tt0.pbcor_inset.pdf}
{figures/Orion_SourceI_B6_continuum_r-2.clean0.1mJy.selfcal.ampphase5.deepmask.image.tt0.pbcor_inset.pdf}
{An overview figure showing the observed region and highlighting some of
the most prominent detected sources.
The main image shows most of the field of view centered on Orion Source I.
The colorbar shows the intensity of this image.
The surrounding hot core is visible as diffuse emission along the northeast-southwest
axis from Source I.
The inset figures show, clockwise from top left, 
Source I (intensity range -1 to 20 mJy \perbeam),
AlmaBNKLBinary1 and AlmaBNKLBinary2 (-0.5 to 2 mJy \perbeam),
Source BN (-1 to 100 mJy \perbeam),
IRC 6E (-1 to 6 mJy \perbeam),
IRC 2C (-1 to 7 mJy \perbeam),
Source N (-1 to 5 mJy \perbeam)
AlmaHotCoreDisk1  (-1 to 4 mJy \perbeam),
and
WMJ053514.797-052230.557 (-1 to 10 mJy \perbeam).
The insets are expanded by 10$\times$ relative to the main image.
The synthesized beam is shown at the same 10$\times$ zoom factor in the
bottom-left; it has contours overlaid at 5, 10, 20, and 30\% of peak
in purple, blue, green, and yellow respectively.
}
{fig:overview}{1}{7.5in}

\subsubsection{Other sources}
We note the detection of several other sources in the field, many of
which are previously undetected.  These will be discussed in a future work,
but we note the presence of an edge-on disk slightly smaller than Source I's
disk embedded within the `hot core' shown in Figure \ref{fig:overview},
which we name AlmaHotCoreDisk1, and a binary separated by 0.22\arcsec (93 AU)
we label AlmaBNKLBinary1 and AlmaBNKLBinary2 ({\color{red} these names are
placeholders}).
