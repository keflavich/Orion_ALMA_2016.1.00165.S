\documentclass[twocolumn]{aastex61}
\newcommand{\Bsixmaj}{0.037}
\newcommand{\Bsixmin}{0.022}
\newcommand{\Bsixpa}{67}
\newcommand{\Bthreemaj}{0.065}
\newcommand{\Bthreemin}{0.041}
\newcommand{\Bthreepa}{50.9}

\input{preface}
\begin{document}

\title{The Mass of Orion's Source I}
\begin{abstract}
   We resolve the disk around source I and measure the rotation curve.  Because
   of the higher resolution of these observations, we were able to measure the
   outer envelope of the rotation curve of the \water $5_{5,0}-6_{4,3}$ line,
   which gives a mass $M_I\approx19$ \msun.  Using centroid-of-channel methods,
   we still infer a higher mass than previous authors, with
   $M_I\gtrsim14$\msun.
   These measurements solidify Source I as a genuine high-mass protostar
   and provide some support for the Source I, BN, Source X dynamical decay
   scenario as an explanation for the explosive molecular outflow.
\end{abstract}

\section{Introduction}
Source I is the closest candidate forming high-mass ($M>8$ \msun) star, 
and as such is the most important protostar for testing basic theories
of how massive stars form.  However, despite its relative proximity at
a mere 414 pc from the sun \citep{Menten2007a}, the mass of Source I
has been the subject of prolonged debate, with several estimates
\citep[e.g.][]{Plambeck2016a} putting
its mass below the classic 8 \msun threshold for a single star to go supernova
\citep[][]{Heger2003}.
% \color{red} Should this citation be qualified?


Several attempts have been made to measure the mass of Orion Source I using the
rotation curve of various molecular lines:
\begin{itemize}
    \item \citet{Hirota2014a} observed vibrationally excited \water emission
        from the $v_2=0, 10_{2,9}-9_{3,6}$ and  $v_2=1, 5_{2,3}-6_{1,6}$ lines
        with $E_L=1846$ and 2939 K, respectively.  They imaged these lines
        with $0.40\arcsec\times0.34\arcsec$ resolution and made
        \textit{velocity centroid maps} of the position of peak intensity
        as a function of velocity to measure the rotationally supported
        mass in Source I.  They obtained a mass estimate of $5-7\msun$.
    \item \citet{Plambeck2016a} measured both the continuum SED and the rotation
        curve of gas around Source I.  They used a centroiding approach
        similar to \citet{Hirota2014a} to measure the rotation curve of
        SO, SiS, SiO, and CO and infer the source mass $M\sim5-7$ \msun.
    \item \citet{Matthews2010a} used 3D measurements of SiO maser velocities
        at 0.5 mas (0.2 AU) resolution along a `bridge' hovering just above and
        below the disk and an assumed radius $r\sim35$ AU  to infer a mass
        $M\approx8-10$ \msun.
\end{itemize}

Source I's mass is also closely tied to the origin of the Orion Outflow.
Several authors argue that Source I, BN, and Source N \citep[or, alternatively,
source X][]{Luhman2017a} were part of a single non-hierarchical multiple system
that underwent dynamical decay, and this decay somehow triggered the outflow
\citep{Bally2005a,Rodriguez2005a,Goddi2010a,Bally2011a,Bally2015a,Bally2017a}.  However, others have
noted that the lower masses inferred for Source I above are incompatible with
this scenario \citep{Chatterjee2012a,Farias2017a,Plambeck2016a}, which requires
a mass $M_{I} \gtrsim 15$ \msun.

\section{Observations}

At 1.3 mm in our Robust 0.5 image, the beam is $0.052\times0.030$ \arcsec at
PA$=-77.7\degrees$ ($21.8\times12.7$ AU), resulting in a 
conversion factor of 15000 K Jy$^{-1}$.

% Source I is directly at the phase center of our observations.  However,
% BN is quite far from the phase center, at the 70\% of peak recovery
% point in Band 6 and 95\% in Band 3.  
% {\color{red}To account for wide-field non-coplanar effects \citep{Cornwell2008a},
% we imaged the full field using CASA \texttt{tclean}'s \texttt{wproject} gridder
% with 30 \texttt{wprojplanes}.}


\begin{table*}[htp]
\centering
\caption{Observation Summary}
\begin{tabular}{llllll}
\label{tab:observations}
Date & Band & Array & Observation Duration &  Baseline Length Range  & \# of antennae\\
     &      &       & seconds              & meters                    & \\
\hline
08-Oct-2016 & 6 & 12m & 1013 & 17-3144 & 43\\
31-Oct-2016 & 7 & 12m & 2671 & 19-1124 & 42\\
19-Sep-2017 & 6 & 12m & 4342 & 41-12147 & 42\\
24-Sep-2017 & 3 & 12m & 5146 & 21-12147 & 41\\
\hline
\end{tabular}
\end{table*}


The continuum images have dynamic range in the vicinity of Source I of about 200,
with local noise in the range 0.1 mJy \perbeam and a measured peak intensity of 17-30
mJy/beam depending on the weighting.
{\color{red} Our target noise level was $\sim20$ \microJy \perbeam,
about five times better.  We received only half the data, so that goes up to $\sim30$
\microJy \perbeam, still a factor of 3 deeper than we achieved.  It *may* be possible
to recover some of this through self-calibration and deeper cleaning, but that
is not so obvious to me; excepting the streaks near Source I, I don't believe
we are dynamic-range-limited.}


\begin{table*}[htp]
\centering
\caption{Continuum Image Parameters}
\begin{tabular}{ccccccccc}
\label{tab:image_metadata}
Band & Robust & Beam Major & Beam Minor & Beam PA               & $T_B$/$S_\nu$      & RMS & Source I $S_{\nu,max}$ & Dynamic Range\\
     &        & \arcsec    & \arcsec    & $\mathrm{{}^{\circ}}$ & $10^3$ K Jy$^{-1}$ & $\mathrm{mJy}~\mathrm{beam}^{-1}$ & $\mathrm{mJy}~\mathrm{beam}^{-1}$ & \\
\hline

B6 & -2 & 0.037 & 0.022 & 67.0 & 30.3 & 0.087 & 19.638 & 220 \\
B3 & -2 & 0.065 & 0.041 & 50.9 & 53.2 & 0.038 & 14.179 & 370 \\

\hline
\end{tabular}

\end{table*}


\Figure
%{figures/Orion_SourceI_B6_continuum_r-2.clean0.5mJy.selfcal.ampphase5.image.tt0.pbcor_inset.pdf}
{figures/Orion_SourceI_B6_continuum_r-2.clean0.1mJy.selfcal.ampphase5.deepmask.image.tt0.pbcor_inset.pdf}
{An overview figure showing the observed region and highlighting some of
the most prominent detected sources.
The main image shows most of the field of view centered on Orion Source I.
The colorbar shows the intensity of this image.
The surrounding hot core is visible as diffuse emission along the northeast-southwest
axis from Source I.
The inset figures show, clockwise from top left, 
Source I (intensity range -1 to 20 mJy \perbeam),
AlmaBNKLBinary1 and AlmaBNKLBinary2 (-0.5 to 2 mJy \perbeam),
Source BN (-1 to 100 mJy \perbeam),
IRC 6E (-1 to 6 mJy \perbeam),
IRC 2C (-1 to 7 mJy \perbeam),
Source N (-1 to 5 mJy \perbeam)
AlmaHotCoreDisk1  (-1 to 4 mJy \perbeam),
and
WMJ053514.797-052230.557 (-1 to 10 mJy \perbeam).
The insets are expanded by 10$\times$ relative to the main image.
The synthesized beam is shown at the same 10$\times$ zoom factor in the
bottom-left; it has contours overlaid at 5, 10, 20, and 30\% of peak
in purple, blue, green, and yellow respectively.
}
{fig:overview}{1}{7.5in}

\section{Results}

\subsection{Continuum}
We detect the disk in the continuum at 3mm, 1mm, and 0.8mm, but we detect
spectral lines only from the surfaces above and below the continuum disk at 1
mm (Figure \ref{fig:continuum_data_B6} shows the 1 mm continuum).  The nondetection of
lines in the disk midplane is a strong indication that the continuum is
optically thick, as has previously been noted
\citep[e.g.][]{Plambeck2016a}.

We fit the highest-resolution 1.3 mm and 3.2 mm continuum image with a simple
model to determine the basic
observational structure.  We used a linear model (i.e., an infinitely thin
perfectly edge-on disk) for the disk, with endpoints and amplitude as free parameters.
This simple model left significant residuals, so we added a two-dimensional
Gaussian smoothing kernel as another three free parameters to obtain a
substantially better fit (see Figure \ref{fig:contmodel_residuals_B6}).
The optimization was performed using a Levenberg-Marquardt
fitter \citep{Newville2014a}. %https://zenodo.org/record/11813#.WlPUoGS7_pQ

We determined that the disk is resolved in both
directions, with a scale height of 20-30 AU and a length of about
90 AU (see Table \ref{tab:continuum_fit_parameters}).  These measurements
are very close to those published by \citet{Plambeck2016a}, though their data
only marginally resolved the source at wavelengths 1 mm and shorter.

This simple model leaves a significant residual pointlike source near
the center of the disk, which we measured by adding a smeared point
source to the model in Figure \ref{fig:contmodel_residuals_B6},
where we have allowed the source to be smeared only in the direction
of the disk, limiting the number of free parameters.
This source is discussed further in Section \ref{sec:ptsrc}.

Table \ref{tab:continuum_fit_parameters} also includes measurements
of the total integrated intensity and the ratio of the pointlike source to
the total.

We recover a total intensity of 290 mJy for the Source I disk
system in our robust -2 image, which is  $\sim80\%$ what
\citet{Plambeck2016a} reported.  We recover a higher integrated intensity of
420-450 mJy - higher than \citet{Plambeck2016a} - when using the robust 0.5 
or robust 2 images.  This higher recovered flux density implies some emission
is resolved out, implying that the Source I spectral energy distribution (SED)
shape is significantly dependent upon interferometer array parameters.
% {\color{red} Or, if we really trust the low-level emission, we can argue
% and fit an ``upper atmosphere'' of the Source I disk.  The elliptical
% residual in Fig \ref{fig:contmodel_residuals_B6} could be real.
% It kind of looks like a disk at inclination $\sim60$, but since it's a
% very faint residual, I wouldn't give that interpretation a second thought.
% Since it is widest near the center, it is not a flared disk.  What
% else could it be?}

\begin{table*}[htp]
\centering
\caption{Continuum Fit Parameters}
\begin{tabular}{cccccccc}
\label{tab:continuum_fit_parameters}
Frequency & Ptsrc Frac & Disk Scaleheight & Total Flux & Ptsrc Position & Disk PA & Ptsrc Amp & Disk Radius \\
$\mathrm{GHz}$ & $\mathrm{}$ & $\mathrm{AU}$ & $\mathrm{mJy}$ &  & $\mathrm{{}^{\circ}}$ & $\mathrm{mJy}$ & $\mathrm{AU}$ \\
\hline
93.3 & 0.065 & 11 & 58 & 05h35m14.5174s -05d22m30.6092s & -39 & 3.7 & 44 \\
224.0 & 0.0052 & 19 & 300 & 05h35m14.5182s -05d22m30.6072s & -39 & 1.6 & 50 \\
\hline
\end{tabular}

\par The centroid coordinate of the point source is specified in ICRS coordinates.
\end{table*}


The disk position angle points to within 2 degrees of the Becklin-Neugebauer
object (Orion BN); the PA of the vector from Source I to Source BN is -37.6
degrees, while the measured disk position angle is -36 to -37 degrees.
This coincidence was noted by \citet{Bally2011a} and is now much more strongly
corroborated.
%There is no remaining doubt that the disk is oriented NW-SE \citep[][contended
%the disk had the opposite orientation]{Testi2010a}.

% This additional
% source may be part of a flared or warped inner disk, a large companion, or part
% of the inner outflow.  Could it also be part of an accretion flow?  It would be
% worth monitoring this blob to see if it is in orbit or infalling or something
% else.  

The disk has a brightness temperature ranging from 200-500 K.  These measurements
agree well with the continuum model of \citet{Plambeck2016a}, who inferred
the presence of an optically thick $T=500$ K surface from the SED.
{\color{red} The region of the surrounding hot core that is in line with the
disk is slightly fainter than its surroundings, hinting that the disk
is shading the core from the central source and adding circumstantial
confirmation that the disk is optically thick and cooler than the central
source.   TODO: this statement isn't particularly clear now; I'm saying
that if you trace a line along the disk direction into the hot core, it is slightly
fainter (at 1mm) than to either side.}

\FigureTwo
{figures/OrionSourceI_data_B6.pdf}
{figures/OrionSourceI_data_stretched_B6.pdf}
{The robust -2 continuum image of Orion Source I at 1.3 mm.  The beam is shown
in the bottom right, with size $\Bsixmaj\arcsec\times \Bsixmin\arcsec$ at
PA$=\Bsixpa\degrees$.
The left panel shows the full intensity range, while the right panel
shows a narrower stretch to highlight some of the extended features
around the disk, which may be imaging artifacts (the synthesized beam
has $\sim30\%$ sidelobes approximately along the East-West axis).
Both panels have flux density and brightness temperature colorbars.
}
{fig:continuum_data_B6}{1}{3.5in}

\FigureTwo
{figures/OrionSourceI_data_B3.pdf}
{figures/OrionSourceI_data_stretched_B3.pdf}
{The robust -2 continuum image of Orion Source I at 3.3 mm.  The beam is shown
in the bottom right, with size $\Bthreemaj\arcsec\times \Bthreemin\arcsec$ at
PA$=\Bthreepa\degrees$.
}
{fig:continuum_data_B3}{1}{3.5in}

\Figure{figures/models_and_residuals_B6.pdf}
{A series of plots showing the band 6 continuum models used and their residuals.
The top row shows the models, starting from a simple 1D linear model convolved
with the beam (left), continuing with a disk smoothed with a broader beam to
account for scale height (middle), and finally a version of the middle model
with a smeared point source added (right).  The fit parameters are given in Table
\ref{tab:continuum_fit_parameters}.  The second and third row show the
residuals (data - model) for each of the models in the top row; the bottom row
uses a narrow linear scale to emphasize the lower-amplitude residuals, while
the top two use an arcsinh stretch to display the full dynamic range.
}
{fig:contmodel_residuals_B6}{1}{7.5in}


\Figure{figures/models_and_residuals_B3.pdf}
{A series of plots showing the band 3 continuum models used and their residuals.
See the caption of Figure \ref{fig:contmodel_residuals_B6} for details.
}
{fig:contmodel_residuals_B3}{1}{7.5in}

% \FigureThree
% {figures/SourceI_Disk_model.png}
% {figures/SourceI_Disk_model_bigbeam.png}
% {figures/SourceI_Disk_model_bigbeam_withptsrc.png}
% {continuum modeling}
% {fig:contmodels}{1}{3in}

\subsubsection{Other sources}
We note the detection of several other sources in the field, many of
which are previously undetected.  These will be discussed in a future work,
but we note the presence of an edge-on disk slightly smaller than Source I's
disk embedded within the `hot core' shown in Figure \ref{fig:overview},
which we name AlmaHotCoreDisk1, and a binary separated by 0.22\arcsec (93 AU)
we label AlmaBNKLBinary1 and AlmaBNKLBinary2 ({\color{red} these names are
placeholders}).


\subsection{Water Line}
The brightest and most interesting line is the \water $5_{5,0}-6_{4,3}$ line at
232.68670 GHz, with $E_U=3461.9$ K.  \citet{Hirota2012a} detected this line
in 2\arcsec resolution ALMA Science Verification data, but believed it to be
masing.  We report here that, because it is similar in morphology and
excitation level to the 336 GHz vibrationally excited water line reported in
\citet{Hirota2014a}, and it has a peak brightness temperature $\sim1000$ K, it
is most likely a thermal line.

The water line traces an X-shaped feature above and below the disk, 
resembling the overall distribution of SiO masers but with a somewhat narrower
distribution.  Since the water is not directly aligned with the continuum
disk, it may not be a direct kinematic tracer of orbits within the disk,
but it is at least as good a tracer as SiO.  Because the water emission is
thermal, though, it exhibits less extreme brightness fluctuations than 
the SiO masers, allowing us to fit an upper-envelope velocity curve in Section
\ref{sec:kinematics}.

% While the continuum disk truncates at r=50 AU, the lines show emission
% beyond this radius, with \water exhibiting emission out to $r\sim100$ AU
% {\color{red} This is a coarse estimate, and the water probably only goes out to
% $r>50$ AU above and below the disk}.

\subsection{Other lines}
Several unidentified lines exhibit emission at the outer edge of the continuum disk.
The peak signal from these lines appears within the $T_B\gtrsim75$ K contour
(Figure \ref{fig:U1peak}).
Their lack of association with the inner disk suggests that they trace the outer
surface of an optically thick disk.
If a molecule is at a similar temperature to the dust, but its line approaches
an optical depth near unity at a lower column density of \hh than the dust,
it should appear in emission.  These lines therefore appear to trace the disk
directly and provide the best direct tools for measuring the disk kinematics.

\FigureTwo
{{figures/OrionSourceI_Unknown_1_robust0.5.maskedclarkclean10000_medsub_K_peak}.png}
{{figures/OrionSourceI_H2Ov2=1_5(5,0)-6(4,3)_robust0.5.maskedclarkclean10000_medsub_K_peak}.png}
{Peak intensity map of an unknown line (U230.321535; left) and the \water
line with continuum overlaid
in contours.  Continuum contours are shown in red at levels of 1, 5, 10, 20, and 30
mJy \perbeam (15, 75, 150, 300, 450 K).
The \water and unknown line clearly trace different physical structures, as
they exhibit no coincident emission peaks.
% {\color{red} Note to coauthors: there is a slight difference in position between the
% center of the continuum emission and the midpoint of the two line emission blobs.
% Do we make anything of this?  It becomes fairly significant in the centroid-of-channel
% analysis in the Appendices.}
}
{fig:U1peak}{1}{3in}

\FigureTwo
{{figures/OrionSourceI_Unknown_1_robust0.5.maskedclarkclean10000_medsub_K_moment0}.png}
{{figures/OrionSourceI_H2Ov2=1_5(5,0)-6(4,3)_robust0.5.maskedclarkclean10000_medsub_K_moment0}.png}
{Moment-0 (integrated intensity) map of the U230.321535 and \water line with continuum overlaid
in contours.  Continuum contours are shown in red at levels of 1, 5, 10, 20, and 30
mJy \perbeam (15, 75, 150, 300, 450 K).
}
{fig:h2omom0}{1}{3in}

\subsection{Kinematics: a Keplerian disk}
\label{sec:kinematics}
Following \citet{Seifried2016a}, we measure the outer edge of the detected
emission to define the rotation curve surrounding Source I.  In Appendix
\ref{appendix:centroids}, to facilitate direct comparison with previous works,
we use the centroid-of-velocity-channel approach that \citet{Seifried2016a}
advise against.

Of the detected lines, only the \water line spans a significant range of
velocities as a function of radius; as shown in Figures \ref{fig:h2okepler},
there is \water emission spanning at least from radius 10 to 100 AU.  Many
other molecules, most of
which we have not been able to identify, span a range of 30-80 AU, while SiS
spans 30 AU to an unconstrained outer radius.  Because the water line,
which traces emission mostly above and below the disk, shows a good kinematic
match to the unknown lines that more directly trace the disk, we argue it is
a good tracer of the disk's kinematics.

We find that a 19 \msun edge-on Keplerian rotation curve fits the outer edge of
the \water line reasonably well (Figure \ref{fig:h2okepler}), and is also an
acceptable match to the outer profiles of other unidentified lines
(Appendix \ref{sec:otherlines}).  A smaller mass, such as
the 5-10 \msun suggested previously \citep{Plambeck2016a,Hirota2014a}, is
inconsistent with the data: for such masses, emission is clearly detected
outside of the Keplerian curve.  These lower masses are therefore excluded.  

\FigureTwo
{figures/H2O_kepler_SeifriedPlot_0.01arcsec.png}
{figures/H2O_kepler_SeifriedPlot_0.1arcsec.png}
{Position-velocity diagram of \water $5_{5,0}-6_{4,3}$.
The blue dotted curve is the outer envelope of the velocity curve
determined using the method of \citet{Seifried2016a}.
Red curves show the Keplerian velocity profile surrounding a 19 \msun
central source; green dotted curves show a Keplerian profile for a 10 \msun
source.
White dashed lines indicate the adopted source central position
J2000 5:35:14.519 -5:22:30.633 (FK5) and central velocity (6 \kms).
The purple dashed lines show the full orbital path for radii of
10 and 100 AU, and indicate the approximate limits of the disk.
The left version is extracted from a range $\pm0.05\arcsec$ (50 AU)
around the fitted centerline of the continuum emission [actually, it's from
a drawn line right now], so it includes significant amounts of `superdisk'
emission.  The right version is extracted from a narrower $\pm0.01\arcsec$
range, less than a beam width, and shows the best approximation of the midplane
available.
{\color{red} NOTE TO SELF: Why are these figures different sizes?  They are from nearly
identical data and \emph{should} be identical shapes.}
{\color{blue} Possible comparison figures to reference:
\citet{Matra2017a} Figure 4 (5 AU resolution, $r\sim150$ AU disk),
\citet{Dutrey2017a} Figure 4 (60 AU resolution, $r\sim250$ AU disk),
others?
}
}
{fig:h2okepler}{1}{3.5in}


The main reason our measurements differ from those of \citet{Plambeck2016a},
\citet{Hirota2014a}, and \citet{Matthews2010a} is that our spatial resolution
is substantially better.  This improved resolution allows us to use a better
method to infer the mass from the rotation curve.  Our method is motivated by
the simulations of \citet{Seifried2016a}, who show that the centroid-based
method used in previous works (which they were forced to use because of their
lower spatial resolution) systematically underestimates the central source
mass.  A detailed comparison of our measurements to those of
\citet{Plambeck2016a} is given in Appendix \ref{appendix:centroids}, in which
we conclude that, even using the deprecated method, we infer a mass $M_I \sim
14$ \msun.

% {\color{red} Note to coauthors:
% What uncertainties are there in these measurements?
% Do we need to account for turbulent line broadening?
% \citet{Flaherty2017a} suggest that there is negligible turbulence in low-mass disks,
% but what about Source I?
% 
% If there's no turbulence, and we assume $T\sim1000$ K, the
% linewidth of \hh is 2.7 \kms, which is enough to lead us to overestimate the mass by a
% non-negligible amount.  Do we need to correct for this by subtracting off (2.7 / 2) \kms
% from the line extrema?  Or, do we assume that the molecules we are detecting have
% much narrower thermal broadening because they come from massive molecules, e.g., \methanol,
% such that the linewidths are $<1 \kms$?
% }


\section{Discussion}
\subsection{Mass of Source I}
We measure a mass of $M_I=19$ \msun, which is higher than most measurements
previously reported.  Our mass measurement is higher than previous works
because our spatial resolution is high enough to allow a direct fit of the
rotation curve to the outer envelope of an emission line in position-velocity
space.

This mass measurement confirms that Orion is still a region with ongoing
high-mass star formation.  If Source I were below $M<8$ \msun, Orion might
instead have been regarded as a region of past high-mass star formation.

\subsection{The luminosity of Source I}
Since we observe an optically thick surface, we can infer the luminosity
required to keep such a surface at the observed $\approx$500 K assuming
it is heated only by radiation.  Taking the disk radius to be 50 AU,
the required central source luminosity is 6500 \lsun.  This estimate
should be taken as a lower limit, since the inner disk is likely to be
optically thicker and capable of shielding the outer disk, thereby
keeping the observed $\tau=1$ surface at 1 mm cooler than would
be produced by radiative equilibrium.

\subsection{The dynamical decay scenario}
{\color{red} Should this paragraph be left in the intro?}
\citet[][and earlier authors?]{Bally2011a} suggested that the high proper
motion of Source I, BN, and Source n, combined with the observed \hh outflow,
implied the outflow and the runaway stars were produced in the same single
event $\sim500$ years ago.  That event was the dynamical decay of a
non-hierarchical multiple system, i.e., it was the interaction of multiple
stars at the center of a mini cluster.  More recent observations by
\citet{Luhman2017a} have shown that Source n is unlikely to have participated
in this interaction, but instead that Source X, another star within the same
field, has high proper motion that points back to the interaction center.

\citet{Farias2017a} report that, while any dynamical decay scenario involving
Sources I, X, and BN that can reproduce the observed proper motions are
unlikely, those with a higher mass for source I ($M_I>14$ \msun) are the only
ones capable of producing the observed proper motions.  Our observed higher
mass for source I therefore keeps the dynamical interaction scenario viable.

\subsection{The `flyby' scenario}
\citet{Tan2004a} and \citet{Chatterjee2012a} described a scenario in which BN
was ejected from the $\theta$1C system and flew through the Orion Hot Core that
contains Source I.   Our observations provide one important constraint on this model:
the disk position angle.
The Source I disk position angle is  closely aligned to the proper
motion vector of BN and the vector connecting Source I and BN,
implying that BN played a significant role in the disk's formation.
This role in turn implies that BN must have come very close to Source I,
within $<100$ AU, and greatly reduces the interaction cross-section
from a $\theta$1C ejection.


\subsection{Is the disk consistent with the dynamical ejection model?}
\citet{Plambeck2016a} argue that both the mass of Source I and the presence of
the disk rule out the dynamical ejection model of \citet{Bally2011a}.  We show
here that the star is significantly more massive, but what about the disk?

Following \citet{Bally2011a}, we note that the disk truncates at $R<50$
AU.  At this radius, the orbital timescale is $\sim100$ years, so gas at the
disk's outer radius would have had five dynamical times to relax into a
circular disk configuration after the explosive event.

The alignment of Source I's disk with the I-BN vector is consistent with
a dynamical interaction between these sources.  If the ejection resulted
in Source I and BN being launched in nearly opposity directions from their
center of mass (which must have been moving in the rest frame of the Orion
nebula; see Bally+ in prep), any material around Source I that remained
bound would be dragged in the direction of Source I, and would therefore
have an angular momentum vector orthogonal to the direction of motion.



\subsection{The pointlike source in the disk}
\label{sec:ptsrc}
We have clearly detected a compact (but marginally resolved) source near the
center of the disk at both 3mm and 1mm.  The sources has a flux rising from 1
mm to 3 mm - behavior generally inconsistent with a thermal source.  However, since the
source is nearly coincident with an edge-on disk that we show is optically
thick at 1 mm, it is likely that the source is thermal but is significantly
attenuated by the disk at 1 mm and seen clearly at 3 mm.

\citet{Reid2007a} used comparable-resolution 7 mm VLA data to infer
the presence of a 2.2 mJy source at the center of the Source I disk.
The pointsource-to-disk flux ratio at 7 mm was $\sim20\%$, substantially
higher than we observe at 1 mm but comparable to our
observation at 3 mm (Table \ref{tab:continuum_fit_parameters}).  
% -log(2.2/10) / log(7/3)
The spectral index from 3 to 7 mm is $\alpha=1.8$, close to an optically
thick blackbody, suggesting that the disk is nearly optically thin at 3 mm.

The central source has a surface temperature $T\geq1250$ K, the brightness
temperature of a 2.2 mJy source within a $41\times28$ milliarcsecond beam at 7
mm.  If the source is a 5000 K blackbody \citep[e.g.,][]{Testi2010a}, it must
have a radius $R=7.5$ AU.  Such a gigantic star is implausible, as it would
produce a luminosity of 1.5\ee{6} \lsun, several orders of magnitude higher
than the total luminosity in the region.

What is the emission mechanism from this central source?
It could simply be hot, optically thick dust that is partly obscured by the
cooler disk at higher frequencies.  The  extension of this `source'
along the disk direction in Figure \ref{fig:contmodel_residuals_B6}
suggests that we are seeing the hot inner disk.
As pointed out by \citet{Plambeck2016a}, it is quite unlikely to be
free-free emission, since there are no detected recombination lines.

The source is slightly offset from the center of the disk by 3-5 AU in
projection.  This offset hints that it is not a single central source, but
instead is a hot region of the inner disk.  Such an asymmetry in the disk could
be driven either by instability in the disk or, if the central star is a
binary, by the proximity of the more luminous companion.

\input{solobib}

\appendix
\section{Other unidentified lines with interesting morphology}
\label{sec:otherlines}
In this section, we show two of the other unidentified lines that appear to trace
the disk.  These figures (\ref{fig:U4} and \ref{fig:U5}) show that the lines
are nearly absent from the continuum-bright disk, but are detected along the upper and
lower surfaces and at the extrema along the disk.  The associated position-velocity diagrams

\FigureThree
{{figures/OrionSourceI_Unknown_4_robust0.5.maskedclarkclean10000_medsub_K_moment0}.png}
{{figures/OrionSourceI_Unknown_4_robust0.5.maskedclarkclean10000_medsub_K_peak}.png}
{{figures/keplercurves_sourceI_Unknown_4_robust0.5_diskpv_0.01}.png}
{Moment 0 and peak intensity map of U232.511, similar to Figures \ref{fig:U1peak} and \ref{fig:h2omom0}.
The rightmost panel shows a position-velocity diagram extracted from the disk midplane.
The overlaid  curve shows the Keplerian velocity profiles for a 19 \msun central source in red.
The dashed magenta lines show the velocity curves at r=30 and 80 AU.
%{\color{red} Note to coauthors: Any asymmetry in the velocity center is likely due to an inaccurate
%guess at the unidentified line's rest velocity.}
}
{fig:U4}{1}{2.25in}

\FigureThree
{{figures/OrionSourceI_Unknown_5_robust0.5.maskedclarkclean10000_medsub_K_moment0}.png}
{{figures/OrionSourceI_Unknown_5_robust0.5.maskedclarkclean10000_medsub_K_peak}.png}
{{figures/keplercurves_sourceI_Unknown_5_robust0.5_diskpv_0.01}.png}
{Moment 0 and peak intensity map of U217.9802311, similar to Figures \ref{fig:U1peak} and \ref{fig:h2omom0}.
The rightmost panel shows a position-velocity diagram extracted from the disk midplane.
The overlaid  curve shows the Keplerian velocity profiles for a 19 \msun central source in red.
The dashed magenta lines show the velocity curves at r=30 and 80 AU.
%{\color{red} Note to coauthors: Any asymmetry in the velocity center is likely due to an inaccurate
%guess at the unidentified line's rest velocity.}
}
{fig:U5}{1}{2.25in}

\section{Methodological comparison}
\label{appendix:centroids}
To compare fairly with \citet{Plambeck2016a}, we also used the
centroid-velocity method.  In this approach, we fit two-dimensional Gaussian
profiles to each `blob' in each velocity channel in the PPV cubes of spectral
lines.  Unlike previous works, we have had to fit multiple Gaussians in several
channels, since we resolve the structure and see `blobs' both above and below
the disk.
Figures \ref{fig:velcen_U4}, \ref{fig:velcen_sio}, and \ref{fig:velcen_h2o}
show the results of this analysis.

% {\color{red} Question to coauthors: How should I implement this modeling?
% My current implementation lives at
% \url{https://github.com/keflavich/Orion_ALMA_2016.1.00165.S/blob/master/analysis/edge_on_ring_velocity_model.py},
% and effectively assumes an infinitely small beam.  A larger effective beam will
% have some effect on the observed velocity profile; I believe it will weight
% toward higher velocities, therefore steepening the slope.  I don't incorporate
% line width at all in this analysis, but I can't convince myself it should have
% any effect under the assumption that the disk is symmetric and we are measuring
% centroids.}


We have modeled the velocity profile assuming an edge-on, uniform, optically
thin disk with a sharp central hole and outer truncation.  The
position-velocity curves derive with this approach are shown in the above
figures for a 14 \msun, $30 < R < 70$ AU model, and in Figure \ref{fig:massradiusdemo}
for other masses and radii.  This model approach is the same used by
\citet{Plambeck2016a}.

We compared our data to the models adopted by \citet{Plambeck2016a},
specifically, the 5 and 10 \msun, $20 < R < 50$ AU models.  Neither of these
reproduce the position-velocity curves we have measured with our
higher-resolution data.  We were able to obtain a reasonable match
to the data with the 14 \msun, $30 < R < 70$ AU  model, but this model
may not represent a unique solution.

% While the
% lower-mass source agrees better with the tilt of the velocity curve, it fails
% to reproduce the high observed maximum velocities.

In the edge-on disk models, different inner-radius cutoffs have an effect on
the inner velocity profile slope similar to changing the central mass, so it is
likely that the disk parameters, rather than the central source mass, dominate
our uncertainties in this approach.  
Figure \ref{fig:massradiusdemo} demonstrates this effect: the inner slope of
a 5 \msun, $20~\mathrm{AU} < r < 50~\mathrm{AU}$ disk is indistinguishable
from a 20 \msun , $30~\mathrm{AU} < r < 80~\mathrm{AU}$ disk, though the latter
extends to higher velocity and radius.

The most important conclusion from this modeling is that the higher-resolution
data yield a higher mass than the original lower-resolution data, favoring
a central star with $M>10$ \msun.  However, the centroid-based approach
still appears consistent with a lower mass ($M\lesssim15\msun$) central star
than the \citet{Seifried2016a} upper-envelope fitting approach.  This remaining
ambiguity highlights the difficulty of direct mass inference from assumed
Keplerian velocity curves and the need for more sophisticated disk radiative
transfer models.

\Figure
{figures/Unknown_4_pp_pv_plots.pdf}
{Results of the centroid-velocity analysis for the U232.511 line.
The left panel shows the locations of fitted centroids of the U232.511 line in
the position-position plane relative to the location of Source I as determined
from the line data (5:35:14.517965 -5:22:30.633931 ICRS; note that this is
different from the continuum-determined location).  Source I's position is
marked with a grey circle at the center.  The grey line indicates the disk
midplane as determined from the continuum modeling; it is notably offset from
the line positions.  The circles are colored by their velocity as indicated in
the right panel.  The right panel shows a position-velocity diagram of these
same centroids.  The dotted external curves show Keplerian velocity profiles
for a 5 (red), 10 (green), and 20 (blue) \msun central source.  The black
curve shows the predicted centroid velocity profile of a 14 \msun  edge-on disk
with inner radius 30 AU and outer radius 70 AU.
This model appears to fit this line's centroids reasonably well.
}
{fig:velcen_U4}{1}{6in}

\Figure
{figures/SiOv=1_5-4_pp_pv_plots.pdf}
{Results of the centroid-velocity analysis for the SiO v=1 J=5-4 line.
See Figure \ref{fig:velcen_U4} for details.
Note that the data are quite inconsistent with the disk model;
the SiO emission fitted here traces the bottom of the outflow
and possibly some components of the disk upper atmosphere.
}
{fig:velcen_sio}{1}{6in}

\Figure
{figures/H2Ov2=1_5(5,0)-6(4,3)_pp_pv_plots.pdf}
{Results of the centroid-velocity analysis for the \water line.
Note that the centroid positions do not extend as far from
the source center as the U232.511 line; this effect is likely
a symptom of the centroid-based approach, since the \water
line can be seen extending at least as far as this line.
}
{fig:velcen_h2o}{1}{6in}

\Figure{figures/radius_mass_demo.pdf}
{Plots of the predicted centroid rotation curves for different masses
and inner and outer radial cutoffs.}
{fig:massradiusdemo}{1}{6in}

% \section{Inset figure of the 3 mm data}

\end{document}
