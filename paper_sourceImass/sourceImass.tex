Title: The Mass of Orion's Source I

\section{Introduction}

Several attempts have been made to measure the mass of Orion Source I using the
rotation curve of various molecular lines:
\begin{\itemize}
    \item \citet{Hirota2014a} observed vibrationally excited \water emission
        from the $v_2=0, 10_{2,9}-9_{3,6}$ and  $v_2=1, 5_{2,3}-6_{1,6}$ lines
        with $E_L=1846$ and 2939 K, respectively.  They imaged these lines
        with $0.40\arcsec\times0.34\arcsec$ resolution and made
        \textit{velocity centroid maps} of the position of peak intensity
        as a function of velocity to measure the rotationally supported
        mass in Source I.  They obtained a mass estimate of $5-7\msun$.
    \item \citet{Plambeck2016a}
    \item \citet{Matthews2010a}
\end{itemize}

\section{Results}
We detect the disk in the continuum at 3mm, 1mm, and 0.8mm, but we detect
spectral lines only from the surfaces above and below the continuum disk at 1
mm.  The nondetection of lines in the disk midplane is a strong indication
that the continuum is optically thick.

We fit the continuum image with a simple model to determine the basic
observational structure.  We used a linear model (i.e., an infinitely thin
perfectly edge-on disk) for the disk, with endpoints, amplitude, and smoothing
kernel as free parameters.  We determined that the disk is resolved in both
directions, with a scale height of 0.076\arcsec (31.5 AU) and a length of 0.223
\arcsec (92.5 AU), resulting in a radius $r_{disk}=46.3$ AU.

This simple model leaves a significant 2.5 mJy residual pointlike source at 
05h35m14.5182s -05d22m30.6084s.  This additional source may be part of a flared
inner disk, a companion (unlikely), or part of the inner outflow.  It could
also be part of an accretion flow?  It would be worth monitoring this blob
to see if it is in orbit or infalling or something else.

The disk has a brightness temperature ranging from 200-500 K.

The brightest and most interesting line is the \water $5_{5,0}-6_{4,3}$ line at
232.68670 GHz, with $E_U=3461.9$ K.  \citet{Hirota2012a} detected this line
in 2\arcsec resolution ALMA Science Verification data, but believed it to be masing.
We report here that it is most likely a thermal line, similar to the 336 GHz vibrationally
excited water line reported in \citet{Hirota2014a}, with peak brightness
temperature $\sim1000$ K.



\section{Discussion}
We measure a mass of XXX, which is higher than reported by Hirota, Matthews, etc.

Reasons:
1. Higher frequency (336 GHz, for example) is more optically thick, traces less
of the disk.
2. Method: Siefried showed that vel-of-highest-channel is better than centroid-based
approaches
