\documentclass[twocolumn]{aastex61}
\input{preface}
\begin{document}

\title{The Mass of Orion's Source I}
\begin{abstract}
    Abstract:
    1. We resolve the disk
    2. We measure the rotation curve
\end{abstract}

\section{Introduction}

Several attempts have been made to measure the mass of Orion Source I using the
rotation curve of various molecular lines:
\begin{itemize}
    \item \citet{Hirota2014a} observed vibrationally excited \water emission
        from the $v_2=0, 10_{2,9}-9_{3,6}$ and  $v_2=1, 5_{2,3}-6_{1,6}$ lines
        with $E_L=1846$ and 2939 K, respectively.  They imaged these lines
        with $0.40\arcsec\times0.34\arcsec$ resolution and made
        \textit{velocity centroid maps} of the position of peak intensity
        as a function of velocity to measure the rotationally supported
        mass in Source I.  They obtained a mass estimate of $5-7\msun$.
    \item \citet{Plambeck2016a} measured both the continuum SED and the rotation
        curve of gas around Source I.  They used a centroiding approach
        similar to \citet{Hirota2014a} to measure the rotation curve of
        SO, SiS, SiO, and CO and infer the source mass $M\sim5-7$ \msun.
    \item \citet{Matthews2010a} used 3D measurements of SiO maser velocities
        at 0.5 mas (0.2 AU) resolution along a `bridge' hovering just above and
        below the disk and an assumed radius $r\sim35$ AU  to infer a mass
        $M\approx8-10$ \msun.
\end{itemize}

\section{Observations}

At 1.3 mm in our Robust 0.5 image, the beam is $0.052\times0.030$ \arcsec at
PA$=-77.7\degrees$ ($21.8\times12.7$ AU), resulting in a 
conversion factor of 15000 K Jy$^{-1}$.

\section{Results}

\subsection{Continuum}
We detect the disk in the continuum at 3mm, 1mm, and 0.8mm, but we detect
spectral lines only from the surfaces above and below the continuum disk at 1
mm.  The nondetection of lines in the disk midplane is a strong indication
that the continuum is optically thick.

We fit the 1.3 mm continuum image with a simple model to determine the basic
observational structure.  We used a linear model (i.e., an infinitely thin
perfectly edge-on disk) for the disk, with endpoints, amplitude, and smoothing
kernel as free parameters.  We determined that the disk is resolved in both
directions, with a scale height of 0.076\arcsec (31.5 AU) and a length of 0.223
\arcsec (92.5 AU), resulting in a radius $r_{disk}=46.3$ AU.  These measurements
are very close to those published by \citet{Plambeck2016a}, though their data
only marginally resolved the source at wavelengths 1 mm and shorter.

This simple model leaves a significant 2.5 mJy residual pointlike source at 
J2000 05h35m14.5182s -05d22m30.6084s.  This additional source may be part of a flared
inner disk, a companion (unlikely), or part of the inner outflow.  It could
also be part of an accretion flow?  It would be worth monitoring this blob
to see if it is in orbit or infalling or something else.

The disk has a brightness temperature ranging from 200-500 K.  These measurements
agree well with the continuum model of \citet{Plambeck2016a}, who inferred
the presence of an optically thick $T=500$ K surface from the SED.

\FigureTwo
{figures/OrionSourceI_data.png}
{figures/OrionSourceI_data_stretched.png}
{The continuum image of Orion Source I at 1.3 mm.  The beam is shown in the
bottom right, with size $0.052\arcsec\times0.030\arcsec$ at PA$=103\degrees$.
The left panel shows the full intensity range, while the right panel
shows a narrower stretch to highlight some of the extended features
around the disk, which may be imaging artifacts {\color{red}TODO: might
be able to clean this up a bit, but it might not be worth it for this paper
because it could require indefinite commitment.}.
Both panels have flux density and brightness temperature colorbars.
}
{fig:continuum_data}{1}{3.5in}

\Figure{figures/models_and_residuals.png}
{A series of plots showing the models used and their residuals.
The top row shows the models, starting from a simple 1D linear model
convolved with the beam (left), continuing with a disk with a 0.076\arcsec (31.5 AU)
scale height and a 0.038\arcsec (16 AU) smoothing width (middle), and finally
a version of the middle model with a point source added at 
J2000 05h35m14.5182s -05d22m30.6084s with amplitude 2.5 mJy (39 K).
The second and third row show the residuals (data - model) for each of the models
in the top row; the bottom row uses a narrow linear scale to emphasize the lower-amplitude
residuals, while the top two use an arcsinh stretch to display the full dynamic range.
}
{fig:contmodel_residuals}{1}{7.5in}

% \FigureThree
% {figures/SourceI_Disk_model.png}
% {figures/SourceI_Disk_model_bigbeam.png}
% {figures/SourceI_Disk_model_bigbeam_withptsrc.png}
% {continuum modeling}
% {fig:contmodels}{1}{3in}


\subsection{Water Line}
The brightest and most interesting line is the \water $5_{5,0}-6_{4,3}$ line at
232.68670 GHz, with $E_U=3461.9$ K.  \citet{Hirota2012a} detected this line
in 2\arcsec resolution ALMA Science Verification data, but believed it to be
masing.  We report here that, because it is similar in morphology and
excitation level to the 336 GHz vibrationally excited water line reported in
\citet{Hirota2014a}, and it has a peak brightness temperature $\sim1000$ K, it
is most likely a thermal line.

The water line traces an X-shaped feature above and below the disk, closely
resembling the overall distribution of SiO masers.

While the continuum disk truncates at r=50 AU, the lines show emission
beyond this radius, with \water exhibiting emission out to $r\sim100$ AU.

\subsection{Other lines}
Several unidentified lines exhibit emission at the outer edge of the continuum disk.
The peak signal from these lines appears within the $T_B\gtrsim75$ K contour
(Figure \ref{fig:U1peak}).
Their lack of association with the inner disk suggests that they trace the outer
surface of an optically thick disk.
If the lines are at a similar temperature, but approach an optical depth near unity
at a lower column density of \hh than the dust, they should appear in emission.

\FigureTwo
{{figures/OrionSourceI_Unknown_1_robust0.5.maskedclarkclean10000_medsub_K_peak}.png}
{{figures/OrionSourceI_H2Ov2=1_5(5,0)-6(4,3)_robust0.5.maskedclarkclean10000_medsub_K_peak}.png}
{Peak intensity map of an unknown line (U230.321535; left) and the \water
line with continuum overlaid
in contours.  Continuum contours are shown in red at levels of 1, 5, 10, 20, and 30
mJy \perbeam (15, 75, 150, 300, 450 K).
The \water and unknown line clearly trace different physical structures, as
they exhibit no coincident emission.
}
{fig:U1peak}{1}{3in}

\FigureTwo
{{figures/OrionSourceI_Unknown_1_robust0.5.maskedclarkclean10000_medsub_K_moment0}.png}
{{figures/OrionSourceI_H2Ov2=1_5(5,0)-6(4,3)_robust0.5.maskedclarkclean10000_medsub_K_moment0}.png}
{Moment-0 (integrated intensity) map of the U230.321535 and \water line with continuum overlaid
in contours.  Continuum contours are shown in red at levels of 1, 5, 10, 20, and 30
mJy \perbeam (15, 75, 150, 300, 450 K).
}
{fig:h2omom0}{1}{3in}

\subsection{Kinematics: a Keplerian disk}
Following \citet{Seifried2016a}, we measure the outer edge of the detected
emission to define the rotation curve surrounding Source I.

Of the detected lines, only the \water line spans a significant range of
velocities as a function of radius; as shown in Figures \ref{fig:h2okepler},
there is \water emission spanning at least from radius 10 to 100 AU.  Many
other molecules, most of
which we have not been able to identify, span a range of 30-80 AU, while SiS
spans 30 AU to an unconstrained outer radius.

We find that a 19 \msun edge-on Keplerian rotation curve fits the outer edge
of the \water line reasonably well, and is also an acceptable match to
the outer profiles of other lines.  A smaller mass, such as the 5-10 \msun
suggested previously \citep{Plambeck2016a,Hirota2014a}, is well within the
regime where emission is detected outside of the Keplerian curve.  These lower
masses are therefore excluded.

\FigureTwo
{figures/H2O_kepler_SeifriedPlot_0.01arcsec.png}
{figures/H2O_kepler_SeifriedPlot_0.1arcsec.png}
{Position-velocity diagram of \water $5_{5,0}-6_{4,3}$.
The blue dotted curve is the outer envelope of the velocity curve
determined using the method of \citet{Seifried2016a}.
Red curves show the Keplerian velocity profile surrounding a 19 \msun
central source; green dotted curves show a Keplerian profile for a 10 \msun
source.
White dashed lines indicate the adopted source central position
J2000 5:35:14.519 -5:22:30.633 (FK5) and central velocity (6 \kms).
The purple dashed lines show the full orbital path for radii of
10 and 100 AU, and indicate the approximate limits of the disk.
The left version is extracted from a range $\pm0.05\arcsec$ (50 AU)
around the fitted centerline of the continuum emission [actually, it's from
a drawn line right now], so it includes significant amounts of `superdisk'
emission.  The right version is extracted from a narrower $\pm0.01\arcsec$
range, less than a beam width, and shows the best approximation of the midplane
available.
{\color{red} Why are these figures different sizes?  They are from nearly
identical data.}
}
{fig:h2okepler}{1}{3.5in}


There are two reasons our measurements differ from those of \citet{Plambeck2016a},
\citet{Hirota2014a}, and \citet{Matthews2010a}: first is our substantially better
spatial resolution, and second is the method we use to infer the mass from the
rotation curve.  Our method is motivated by the simulations of \citet{Seifried2016a},
who show that the centroid-based method used by the other authors systematically
underestimates the central source mass.

what about turbulence?  Other authors have argued that there should be unresolved
supersonic turbulence.  If there's not, and we assume $T\sim1000$ K, the
linewidth is 2.7 \kms, which is enough to lead us to overestimate the mass by a
non-negligible amount...  {\color{red} How can we best correct for the thermal
(and maybe turbulent) linewidth?  Subtract 1/2 of the sigma-width from the extrema?
Or perhaps subtract the HWHM?}


\section{Discussion}
We measure a mass of XXX, which is higher than reported by Hirota, Matthews, etc.

Reasons:
1. Higher frequency (336 GHz, for example) is more optically thick, traces less
of the disk.
2. Method: Siefried showed that vel-of-highest-channel is better than centroid-based
approaches

\subsection{Is this disk consistent with the dynamical ejection model?}
\citet{Plambeck2016a} argue that both the mass of Source I and the presence of the disk
rule out the dynamical ejection model of \citet{Bally2011a}.  We show here that
the star is significantly more massive, but what about the disk?

First, we note that the disk truncates at $R<50$ AU.  At this radius, the
orbital timescale is $\sim100$ years, so gas at the disk's outer radius would
have had five dynamical times to relax into a circular disk configuration after the
explosive event.

\input{solobib}

\appendix
\section{Other unidentified lines with interesting morphology}

\FigureThree
{{figures/OrionSourceI_Unknown_4_robust0.5.maskedclarkclean10000_medsub_K_moment0}.png}
{{figures/OrionSourceI_Unknown_4_robust0.5.maskedclarkclean10000_medsub_K_peak}.png}
{{figures/keplercurves_sourceI_Unknown_4_robust0.5_diskpv_0.01}.png}
{Moment 0 and peak intensity map of U232.511, similar to Figures \ref{fig:U1peak} and \ref{fig:h2omom0}.
The rightmost panel shows a position-velocity diagram extracted from the disk midplane.
The overlaid dotted curves show Keplerian velocity profiles for 5 (red), 10 (green), 20 (blue) \msun.
The dashed magenta lines show the velocity curves at r=30 and 80 AU.
}
{fig:U4}{1}{2.25in}

\FigureThree
{{figures/OrionSourceI_Unknown_5_robust0.5.maskedclarkclean10000_medsub_K_moment0}.png}
{{figures/OrionSourceI_Unknown_5_robust0.5.maskedclarkclean10000_medsub_K_peak}.png}
{{figures/keplercurves_sourceI_Unknown_5_robust0.5_diskpv_0.01}.png}
{Moment 0 and peak intensity map of U217.9802311, similar to Figures \ref{fig:U1peak} and \ref{fig:h2omom0}.
The rightmost panel shows a position-velocity diagram extracted from the disk midplane.
The overlaid dotted curves show Keplerian velocity profiles for 5 (red), 10 (green), 20 (blue) \msun.
The dashed magenta lines show the velocity curves at r=30 and 80 AU.
}
{fig:U5}{1}{2.25in}

\end{document}
